Da quando nel 2007 Apple ha presentato al mondo il primo iPhone e Google, sempre nello stesso anno, ha annunciato il suo progetto open source Android, a livello globale si assiste ad una massiccia crescita nella diffusione dei dispositivi mobili.
\\
Da allora sono stati progettati e prodotti dispositivi mobili sempre pi� avanzati, che sono entrati prepotentemente nella vita quotidiana di milioni di persone.
\\
La caratteristica peculiare di questi device � quella di poter offrire all'utente connessione internet in mobilit�. L'attuale generazione di dispositivi mobili consente tale connessione attraverso differenti tipologie di tecnologie wireless: la telefonia mobile ( ad esempio 3G o il pi� recente, e gi� largamente diffuso, standard 4G ) e la tecnologia Wi-Fi.
\\
Contestualmente alla diffusione sempre pi� ampia dei mobile device si sono accompagnati, da una parte, un abbassamento delle tariffe dei piani dati forniti dai gestori telefonici e, dall?altra, una crescita nella presenza di reti Wi-Fi sia pubbliche che domestiche.
\\
L?insieme di questi fattori ha aperto una sterminata gamma di opportunit� per gli sviluppatori, che possono creare app sempre pi� sofisticate e complesse per migliorare la qualit� della vita dell'utente.
\\
Una particolare categoria di applicazioni sono le applicazioni \emph{real-time}. � sufficiente sfogliare l'offerta di un qualsiasi store per applicazioni per accorgersi della vasta offerta di app che consentono di effettuare chiamate o video-chiamate.
\\
Tutte le applicazioni che offrono servizi di interazione multimediale in real-time sono solitamente sviluppate sfruttando il protocollo di livello trasporto UDP.
\\
\\
All'interno del dipartimento di Informatica dell'Universit� di Bologna � stata progettata un'architettura che consente, ad un nodo mobile, di sfruttare contemporaneamente tutte le sue interfacce di rete. Questa architettura, denominata Always Best Packet Switching ( ABPS ), consente, in particolare, di far s� che un'applicazione possa trasmettere il suo flusso informativo attraverso l'interfaccia di rete pi� conveniente, in termini di tempi di risposta e carico della rete, tra quelle disponibili in quel determinato momento.
\\
L'architettura Always Best Packet Switching � composta da:
\begin{itemize}
\item	Un fixed proxy server che gestisce e ri-direziona il flusso informativo dal nodo mobile verso l'esterno e viceversa.
\item Un proxy client in esecuzione sul device mobile che si occupa di mantenere attiva una connessione verso il fixed proxy server per ciascuna interfaccia di rete del nodo mobile.
\end{itemize}
Per valutare qual � l'interfaccia di rete pi� conveniente in un certo istante l'ABPS proxy client sfrutta un modulo chiamato Transmission Error Detector.
\\
Transmission Error Detector si occupa di monitorare ciascun datagram UDP in trasmissione attraverso una specifica interfaccia di rete e di notificare al proxy client l'avvenuta consegna o meno al primo access point. Il modulo Transmission Error Detector � pensato per operare su pi� livelli dello stack di rete del kernel del nodo mobile.
\\
Attraverso Transmission Error Detector si pu� dunque valutare un'eventuale ritrasmissione di un certo datagram UDP attraverso un'altra scheda di rete tra quelle disponibili sul nodo mobile.
\\
Quest'architettura si conf� perfettamente a tipologie di applicazioni come quelle VoIP, le cui principali metriche di valutazione della QoS sono un'alta interazione ed una bassa perdita dei pacchetti UDP.
\\
In particolare, tramite l'architettura ABPS, � possibile mantenere una buona QoS nel caso di datagram persi in fase di invio ( ad esempio a seguito di numerose collisioni ) o nel caso la connessione fornita da una delle interfacce di rete non sia pi� disponibile.
\\
\\
L'obiettivo di questo progetto di tesi � lo sviluppo di un modulo Transmission Error Detector per Wi-Fi compatibile con l'ultima versione del kernel Linux.
\\
Una prima versione del modulo Transmission Error Detector � stata sviluppata per il kernel Linux 2.6.30-rc5. 
\\
%Il porting di Transmission Error Detector verso una versione pi� recente del kernel Linux si rende necessario in ragione della modifica e miglioramento di molti dei moduli di rete che lo compongono.
La necessit� di effettuare il porting di Transmission Error Detector verso una versione pi� recente del kernel Linux nasce dal fatto che molti dei moduli di rete che lo compongono sono stati migliorati e modificati.
\\
Anche il modulo Transmission Error Detector � stato migliorato rispetto alla versione precedentemente sviluppata.
\\
Una delle principali novit� introdotte con la nuova versione � il supporto al protocollo IPv6.
\\
Prima di effettuare il porting si � proceduto a un'attenta fase di studio di come il kernel realizzi effettivamente una comunicazione di tipo UDP a partire dal livello applicativo fino all'interfaccia di rete Wi-Fi. 
\\
Sono state quindi analizzate tutte le strutture dati utilizzate dai moduli di rete del kernel e come esse interagiscono per realizzare una trasmissione lungo la rete.
\\
Si � proceduto quindi con il porting vero e proprio, caratterizzato dall'estensione di ciascun \emph{layer} dello stack di rete coinvolto in una trasmissione di tipo UDP.
\\
Una volta effettuato il porting, � stata definita un' API di sistema in modo tale che una qualsiasi applicazione, in esecuzione su un kernel su cui � installato Transmission Error Detector, possa beneficiare dei meccanismi forniti dal nuovo modulo sviluppato.
\\
Successivamente � stata sviluppata un'applicazione che fa largo uso della nuova interfaccia introdotta, cosicch� fosse possibile testare il nuovo sistema e condurre delle valutazioni sperimentali sui risultati ottenuti.
Successivamente � stata sviluppata un'applicazione che fa largo uso della nuova interfaccia introdotta, cosicch� fosse possibile testare il nuovo sistema  e condurre delle valutazioni sperimentali sui risultati ottenuti.
\\
Questo progetto di tesi � stato sviluppato insieme al collega Alessandro Mengoli che si � occupato principalmente di analizzare i risultati sperimentali frutto di numerosi test effettuati sul sistema montante il modulo Transmission Error Detector.