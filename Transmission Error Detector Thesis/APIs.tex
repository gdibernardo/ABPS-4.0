
\begin{chapter}{Transmission Error Detector developer APIs}
Transmission Error Detector � pensato per operare assieme all'architettura ABPS all'interno di un ABPS proxy client.
\\
Tuttavia una qualsiasi applicazione, in esecuzione su di un kernel a cui � stato installato il modulo TED, pu� essere pensata e implementata per sfruttare il meccanismo di Transmission Error Detection raccontato nel capitolo precedente.
\\
Gli unici accorgimenti che un applicativo deve adottare per beneficiare dei meccanismi forniti dalla versione di Transmission Error Detector sviluppata consistono in:
\begin{itemize}

\item Abilitare il socket UDP utilizzato per la trasmissione alla ricezione degli errori.

\item Adottare la system call sendmsg e specificare nell'ancillary data del messaggio in invio che si � interessati a ricevere un identificativo per quel messaggio oltre che all'indirizzo di memoria dove si vuole venga memorizzato l'identificativo assegnato.

\end{itemize}
\begin{section}{Send a message}
\begin{lstlisting}

/* create IPv4 socket */
int file_descriptor, error, option_value;
file_descriptor = socket(AF_INET, SOCK_DGRAM, 0);

/* Enable socket for error message */
error = setsockopt(*file_descriptor, IPPROTO_IP, IP_RECVERR, (char *)&option_value, sizeof(option_value));

\end{lstlisting}

La system call sendmsg accetta come parametro una struttura di tipo struct msghdr che descrive il contenuto del messaggio in invio.
\\


\begin{lstlisting}

struct msghdr
{
    void         *msg_name;			/* optional address */
    socklen_t     msg_namelen;  		/* size of address */
    struct iovec *msg_iov;        		/* scatter/gather array */
    size_t        msg_iovlen;     		/* # elements in msg_iov */
    void         *msg_control;    		/* ancillary data, see below */
    size_t        msg_controllen; 		/* ancillary data buffer len */
    int           msg_flags;      		/* flags on received message */
};
\end{lstlisting}

Come gi� accennato la struttura struct msghdr pu� contenere delle informazioni di controllo che non saranno trasmesse lungo la rete; questo tipo di informazioni sono chiamate ancillary data.
\\
Gli ancillary data sono una sequenza di strutture di tipo struct cmgshdr.
\\
Gli ancillary data vanno sempre acceduti tramite apposite macro e mai direttamente.
\\
Per accedere alla prima struttura struct cmsghdr di un certo messaggio di tipo struct msghdr, ad esempio, baster� invocare la macro \textbf{CMSG\_FIRSTHDR()} specificando come parametro il puntatore alla struttura msghdr.

\begin{lstlisting}

struct cmsghdr 
{
	socklen_t cmsg_len;    /* data byte count, including header */
	int       cmsg_level;  /* originating protocol */
	int       cmsg_type;   /* protocol-specific type */
	/* followed by unsigned char cmsg_data[]; */
};

\end{lstlisting}

Per specificare che l'app invocante sendmsg � interessata a ricevere l'identificativo da TED � sufficiente settare il campo della struttura struct cmsghdr \textbf{cmgs\_type} al nuovo valore \textbf{ABPS\_CMSG\_TYPE}. 
\\
ABPS\_CMGS\_TYPE � stato definito in socket.h.
\\
All'interno del campo data della struttura struct cmsghdr � necessario copiare il puntatore all'area di memoria in cui si desidera TED setti l'identificativo assegnato al messaggio in invio.
\\
\\
Vediamo un esempio di utilizzo.


\begin{lstlisting}

uint32_t identifier;

uint32_t *pointer_for_identifier = &identifier;

char ancillary_buffer[CMSG_SPACE(sizeof (pointer_for_identifier))];
    
struct iovec iov[3]; /* Scatter-Gather I/O */
    
struct msghdr message_header;
    
struct cmsghdr *cmsg;
    
char buffer[100];
    
memset(buffer,0,100);

strncpy(buffer,"Hello from an app built on top of TED!",100);
    
    
iov[0].iov_base = (void *) buffer; 
iov[0].iov_len = strlen(buffer);
    
    
message_header.msg_name = (void *) &destination_address; /* struct sockaddr_in for destination host */
message_header.msg_namelen = sizeof (destination_address);

message_header.msg_iov = iov;	/* message content*/
message_header.msg_iovlen = 1;
    
message_header.msg_control = ancillary_buffer;
message_header.msg_controllen = sizeof(ancillary_buffer);
    
cmsg = CMSG_FIRSTHDR(&message_header);	/* get first struct cmsg from message_header*/
    
cmsg->cmsg_level = SOL_UDP;
cmsg->cmsg_type = ABPS_CMSG_TYPE;  /* new type for struct cmsghdr defined in socket.h */
    
cmsg->cmsg_len = CMSG_LEN(sizeof(pointer_for_identifier));
    
pointer = (char *) CMSG_DATA(cmsg);	/* accessing cmsg_data field */
    
memcpy(pointer, &pointer_for_identifier, sizeof(pointer_for_identifier));	/* copying pointer to variable in cmsg_data field*/
    
message_header.msg_controllen = cmsg->cmsg_len;
    
    
/* Prepare for sending. */
result_value = sendmsg(file_descriptor, &message_header, MSG_NOSIGNAL);

\end{lstlisting}
\end{section}


\begin{section}{Receive a notification from TED}
Per ricevere una notifica da TED � sufficiente predisporre l'app in lettura sulla coda di errore del socket sfruttato in precedenza per l'invio di un messaggio.
\\
� possibile leggere dalla coda degli errori di un socket attraverso la system call \textbf{recvmsg} specificando il flag MSG\_ERRQUEUE.
\\
La notifica pu� essere ricevuta da un'app allo stesso modo in cui, l'app, potrebbe ricevere un messaggio di tipo ICMP.
\\
Come gi� accennato precedentemente le informazioni della First-hop Transmission Notification sono contenute all'interno di una struttura sock\_extended\_err.


\begin{lstlisting}

/* Fetch the notification from socket error queue */
return_value = recvmsg(file_descriptor, message, MSG_ERRQUEUE | MSG_DONTWAIT);



for(cmsg = CMSG_FIRSTHDR(message); cmsg; cmsg = CMSG_NXTHDR(message, cmsg))
 {
	if((cmsg->cmsg_level == IPPROTO_IPV6) && (cmsg->cmsg_type == IPV6_RECVERR))
	{
		first_hop_transmission_notification = (struct sock_extended_err *) CMSG_DATA(cmsg);
		switch (first_hop_transmission_notification->ee_origin)
	 	{
       			  /* new origin type introduced  */
		 	case SO_EE_ORIGIN_LOCAL_NOTIFY:
                   	{
                        		if(first_hop_transmission_notification->ee_errno == 0)
                        		{			
                        			uint32_t identifier = ted_message_identifier_from_notification(first_hop_transmission_notification);
					printf("Just got a new notification for message marked with identifier \%" PRIu32 ". \n", identifier);
                        			.......
					.......
                        		}
                	   	}
                	   	........
              	     	.......
		}	
	}
}


\end{lstlisting}
\end{section}


\begin{section}{Interact with First-hop Transmission Notification}
Si � deciso di definire una convient API per accedere ai valori memorizzati all'interno della notifica e quindi della struttura sock\_extended\_err.
\\
Le nuove macro introdotte fungono da wrapper per facilitare l'accesso ad alcuni campi della struttura stessa.
\\
L'utilizzo di queste funzioni � consigliato per una maggiore chiarezza e astrazione. Tutte le macro sono definite assieme alla struttura sock\_extended\_err nel file errqueue.h.
\\
Tutte le funzioni definite in questa nuova interfaccia accettano come unico parametro una struttura di tipo sock\_extended\_err.

\begin{paragraph}{ted\_message\_identifier\_from\_notification}
Utilizzata per accedere all'identificativo del messaggio di cui la notifica � associata.
\\
\emph{Parameter:} struct sock\_exetended\_err *notification.
\\
\\
\emph{Return value:} un uint32\_t contenente il valore dell'identificativo del messaggio a cui la notifica si riferisce.
\end{paragraph}
\begin{paragraph}{ted\_message\_status\_from\_notification}
Ritorna lo status di consegna del messaggio a cui la notifica vuole fare riferimento. 
\\
Se il valore ritornato � 1 il messaggio � stato consegnato con successo al primo Access Point ( ACK ), se 0 altrimenti ( NACK ).
\\
\\
\emph{Parameter:} struct sock\_exetended\_err *notification.
\\
\\
\emph{Return value:} un uint8\_t contenente lo status di consegna.
\end{paragraph}
\begin{paragraph}{ted\_message\_retry\_count\_from\_notification}
Ritorna il numero di volte che un messaggio � stato ritrasmesso lungo l'interfaccia di rete prima di essere consegnato.
\\
Un frame marcato come non consegnato ( NACK ) avr� retry count pari a zero.
\\
\\
\emph{Parameter:} struct sock\_exetended\_err *notification.
\\
\\
\emph{Return value:} un uint8\_t contenente il numero di volte che il messaggio � stato trasmesso lungo la rete.
\end{paragraph}
\begin{paragraph}{ted\_message\_fragmentation\_length\_info\_from\_notification}
Restituisce la lunghezza del frammento a cui si riferisce la notifica.
\\
In fase di lettura dalla coda degli errori del socket pi� di una notifica con lo stesso identificativo possono pervenire.
\\
Per disambiguare a quale porzione dati fa riferimento la notifica vengono fornite alcune informazioni aggiuntive sulla frammentazione del messaggio originario.
\\
\\
\emph{Parameter:} struct sock\_exetended\_err *notification.
\\
\\
\emph{Return value:} un uint16\_t contenente la lunghezza espressa in bytes del frammento.
\end{paragraph}
\begin{paragraph}{ted\_message\_fragmentation\_offset\_info\_from\_notification}
Restituisce l'offset, rispetto al frammento originario, del fragment a cui fa riferimento la notifica.
\\
\\
\emph{Parameter:} struct sock\_exetended\_err *notification.
\\
\\
\emph{Return value:} un uint16\_t indicante l'offset del frammento rispetto al frammento originario a cui si riferisce la notifica espressa in bytes del frammento.
\end{paragraph}
\begin{paragraph}{ted\_message\_more\_fragment\_info\_from\_notification}
Ritorna 1 se vi sono altri frammenti relativi allo stesso messaggio, 0 altrimenti.
\\
\\
\emph{Parameter:} struct sock\_exetended\_err *notification.
\\
\\
\emph{Return value:} un uint8\_t indicante la presenza, o meno, di ulteriori frammenti per lo stesso messaggio.
\end{paragraph}
\\
\\
\\
I wrapper per la l'accesso ai campi della stuct sock\_extended\_err appena descritti.

\begin{lstlisting}
/* TED convenient wrapper APIs for First-hop Transmission Notification. */

/* Transmission Error Notification identifier of the datagram whose notification refers to. */
#define ted_message_identifier_from_notification(notification)   ((struct sock_extended_err *) notification)->notification_info

/* Message status to the first hop. Return 1 if the message was successfully delivered to the AP, 0 otherwise. */
#define ted_message_status_from_notification(notification)  ((struct sock_extended_err *) notification)->notification_type

/* Returns the number of times that the packet, associated to the notification provided, was retrasmitted to the AP.  */
#define ted_message_retry_count_from_notification(notification) ((struct sock_extended_err *) notification)->notification_retry_count

/* Returns the fragment length */
#define ted_message_fragmentation_length_info_from_notification(notification)   (((struct sock_extended_err *) notification)->notification_data >> 16)

/* Returns the offset of the current message associated with the notification from the original message. */
#define ted_message_fragmentation_offset_info_from_notification(notification)   ((((struct sock_extended_err *) notification)->notification_data << 16) >> 16)

/* Indicates if there is more fragment with the same TED identifier */
#define ted_message_more_fragment_info_from_notification ((struct sock_extended_err *) notification)->notification_code
\end{lstlisting}


\end{section}

\end{chapter}
