
\begin{chapter}{Transmission Error Detector APIs for developers}
Transmission Error Detector � pensato per operare assieme all'architettura ABPS all'interno di un ABPS proxy client.
\\
Tuttavia una qualsiasi applicazione, in esecuzione su di un kernel a cui � stato installato il modulo TED, pu� essere pensata e implementata per sfruttare il meccanismo di Transmission Error Detection raccontato nel capitolo precedente.
\\
Gli unici accorgimenti che un applicativo deve adottare per beneficiare dei meccanismi forniti dalla versione di Transmission Error Detector sviluppata consistono in:
\begin{itemize}

\item Abilitare il socket UDP utilizzato per la trasmissione alla ricezione degli errori.

\item Adottare la system call sendmsg e specificare nell'ancillary data del messaggio in invio che si � interessati a ricevere un identificativo per quel messaggio oltre che all'indirizzo di memoria dove si vuole venga memorizzato l'identificativo assegnato.

\end{itemize}
\begin{section}{Send a message}
\begin{lstlisting}

/* create IPv4 socket */
int file_descriptor, error, option_value;
file_descriptor = socket(AF_INET, SOCK_DGRAM, 0);

/* Enable socket for error message */
error = setsockopt(*file_descriptor, IPPROTO_IP, IP_RECVERR, (char *)&option_value, sizeof(option_value));

\end{lstlisting}

La system call sendmsg accetta come parametro una struttura di tipo struct msghdr che descrive il contenuto del messaggio in invio.
\\
\begin{lstlisting}

struct msghdr
{
    void         *msg_name;			/* optional address */
    socklen_t     msg_namelen;  		/* size of address */
    struct iovec *msg_iov;        		/* scatter/gather array */
    size_t        msg_iovlen;     		/* # elements in msg_iov */
    void         *msg_control;    		/* ancillary data, see below */
    size_t        msg_controllen; 		/* ancillary data buffer len */
    int           msg_flags;      		/* flags on received message */
};
\end{lstlisting}

Come gi� accennato la struttura struct msghdr pu� contenere delle informazioni di controllo che non saranno trasmesse lungo la rete; questo tipo di informazioni sono chiamate ancillary data.
\\
Gli ancillary data sono una sequenza di strutture di tipo struct cmgshdr.
\\
Gli ancillary data vanno sempre acceduti tramite apposite macro e mai direttamente.
\\
Per accedere alla prima struttura struct cmsghdr di un certo messaggio di tipo struct msghdr, ad esempio, baster� invocare la macro \textbf{CMSG\_FIRSTHDR()} specificando come parametro il puntatore alla struttura msghdr.

\begin{lstlisting}

struct cmsghdr 
{
	socklen_t cmsg_len;    /* data byte count, including header */
	int       cmsg_level;  /* originating protocol */
	int       cmsg_type;   /* protocol-specific type */
	/* followed by unsigned char cmsg_data[]; */
};

\end{lstlisting}

Per specificare che l'app invocante sendmsg � interessata a ricevere l'identificativo da TED � sufficiente settare il campo della struttura struct cmsghdr \textbf{cmgs\_type} al nuovo valore \textbf{ABPS\_CMSG\_TYPE}. 
\\
ABPS\_CMGS\_TYPE � stato definito in socket.h.
\\
All'interno del campo data della struttura struct cmsghdr � necessario copiare il puntatore all'area di memoria in cui si desidera TED setti l'identificativo assegnato al messaggio in invio.
\\
Vediamo un esempio di utilizzo.
\begin{lstlisting}

uint32_t identifier;

uint32_t *pointer_for_identifier = &identifier;

char ancillary_buffer[CMSG_SPACE(sizeof (pointer_for_identifier))];
    
struct iovec iov[3]; /* Scatter-Gather I/O */
    
struct msghdr message_header;
    
struct cmsghdr *cmsg;
    
char buffer[100];
    
memset(buffer,0,100);

strncpy(buffer,"Hello from an app built on top of TED!",100);
    
    
iov[0].iov_base = (void *) buffer; 
iov[0].iov_len = strlen(buffer);
    
    
message_header.msg_name = (void *) &destination_address; /* struct sockaddr_in for destination host */
message_header.msg_namelen = sizeof (destination_address);

message_header.msg_iov = iov;	/* message content*/
message_header.msg_iovlen = 1;
    
message_header.msg_control = ancillary_buffer;
message_header.msg_controllen = sizeof(ancillary_buffer);
    
cmsg = CMSG_FIRSTHDR(&message_header);	/* get first struct cmsg from message_header*/
    
cmsg->cmsg_level = SOL_UDP;
cmsg->cmsg_type = ABPS_CMSG_TYPE;  /* new type for struct cmsghdr defined in socket.h */
    
cmsg->cmsg_len = CMSG_LEN(sizeof(pointer_for_identifier));
    
pointer = (char *) CMSG_DATA(cmsg);	/* accessing cmsg_data field */
    
memcpy(pointer, &pointer_for_identifier, sizeof(pointer_for_identifier));	/* copying pointer to variable in cmsg_data field*/
    
message_header.msg_controllen = cmsg->cmsg_len;
    
    
/* Prepare for sending. */
result_value = sendmsg(file_descriptor, &message_header, MSG_NOSIGNAL);

\end{lstlisting}
\end{section}


\begin{section}{Receive notifications from TED}
Per ricevere una notifica da TED � sufficiente predisporre l'app in lettura sul socket sfruttato in precedenza per l'invio di un messaggio.
\\
� possibile leggere dalla coda degli errori di un socket attraverso la system call \textbf{recvmsg} specificando il flag MSG\_ERRQUEUE.
\\
La notifica pu� essere ricevuta da un'app nello stesso modo l'app potrebbe ricevere un messaggio di tipo ICMP.
\\
Come gi� accennato precedentemente le informazioni della First-hop Transmission Notification sono contenute all'interno di una struttura sock\_extended\_err.
\\
Sono state sviluppate delle APIs per poter accedere a particolari informazioni inerenti la notifica proveniente da TED.
\\



\begin{lstlisting}

\end{lstlisting}

\end{section}

\end{chapter}
