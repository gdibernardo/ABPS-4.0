\documentclass[12pt,a4paper,openright,twoside]{book}
\usepackage[italian]{babel}
\usepackage[latin1]{inputenc}
\usepackage{fancyhdr}
\usepackage{indentfirst}
\usepackage{graphicx}
\usepackage{newlfont}
\usepackage{amssymb}
\usepackage{amsmath}
\usepackage{latexsym}
\usepackage{amsthm}



\begin{document}
\begin{chapter}{TEST E VALUTAZIONI SPERIMENTALI}
Andremo ora a mostrare quali test sono stati effettuati. In particolare mostreremo quali dispositivi sono stati utilizzati per fare le prove, i parametri di valutazione ed i risultati ottenuti.
\\
Lo scopo di questi test � di andare ad osservare la QoS del segnale e come pu� variare in scenari diversi.



\begin{section}{Dispositivi}
Sono stati utilizzati diversi dispositivi. In particolare tre per simulare il client ( o nodo mobile ) ed altri per creare traffico nella rete, per creare condizioni simili ad tipico scenario di utilizzo.

\begin{paragraph}{ZyXEL NBG4615 v2}
Access point utilizzato durante i test. � stata impostata la modalit� Wi-Fi 802.11bgn.

\end{paragraph}



\begin{paragraph}{HP Pavilion dv6 Entertainment PC}
Questo notebook � stato utilizzato come client. Abbiamo montato un kernel versione 4.0.1, modificato tramite la procedura illustrata precedentemente. 
Le specifiche tecniche sono: 
\begin{itemize}
 \item Kernel: Linux versione 4.0.1 modificata
 \item Processore: Intel Core i5 CPU M 430 @ 2.27GHz x 4 
 \item Sistema operativo: Ubuntu 14.04 LTS 64-bit
 
\end{itemize}

\end{paragraph}

\begin{paragraph}{Raspberry Pi Model 2}
Abbiamo utilizzato questo raspberry per creare traffico sulla rete.
Le specifiche tecniche sono: 
\begin{itemize}
 \item Kernel: Linux versione 3.18.0-20-rpi2
 \item Processore: 900MHz quad-core ARM Cortex-A7 CPU 
 \item Sistema operativo: Ubuntu Mate
 
\end{itemize}

\end{paragraph}

\begin{paragraph}{Raspberry Pi Model B}
Abbiamo utilizzato questo raspberry per creare traffico sulla rete.
Le specifiche tecniche sono: 
\begin{itemize}
 \item Kernel: Linux versione 3.18.0-20-rpi2
 \item Processore: 900MHz quad-core ARM Cortex-A7 CPU 
 \item Sistema operativo: Raspbian Wheezy
 
\end{itemize}

\end{paragraph}


\end{section}

\begin{section}{Parametri di valutazione}

abbiamo



\end{section}

\begin{section}{Configurazioni}
Per ottenere dei risultati che potessero rispecchiare un reale utilizzo da parte di un nodo mobile abbiamo creato diverse configurazioni di dispositivi.
In particolare abbiamo combinato l'utilizzo di uno o pi� dispositivi 
\end{section}


\begin{section}{Risultati}

Andiamo ora ad analizzare i risultati ottenuti.
\end{section}




\end{chapter}

\end{document}
