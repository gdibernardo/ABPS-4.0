\documentclass[12pt,a4paper,openright,twoside]{book}
\usepackage[italian]{babel}
\usepackage[latin1]{inputenc}
\usepackage{fancyhdr}
\usepackage{indentfirst}
\usepackage{graphicx}
\usepackage{newlfont}
\usepackage{amssymb}
\usepackage{amsmath}
\usepackage{latexsym}
\usepackage{amsthm}



\begin{document}
\begin{chapter}{TEST E VALUTAZIONI SPERIMENTALI}
Dopo aver illustrato il funzionamento e l'implementazione del TED, passiamo ora ad analizzare i dati relativi alla trasmissione di pacchetti tramite la nostra applicazione.
Lo scopo dei test effettuati � quello di analizzare la QoS del segnale e come pu� variare in diversi scenari.
Questo ci � utile in quanto, in base ai risultati ottenuti, si pu� decidere se cambiare NIC per l'invio di determinati pacchetti.


Andremo ora a mostrare quali test sono stati effettuati. In particolare mostreremo quali dispositivi sono stati utilizzati per fare le prove, i parametri di valutazione ed i risultati ottenuti.
\\
Lo scopo di questi test � di andare ad osservare la QoS del segnale e come pu� variare in scenari diversi.



\begin{section}{Dispositivi}
Per effettuare delle prove sperimentali, sono stati utilizzati diversi dispositivi. In particolare tre per simulare il client ( o nodo mobile ) ed altri per creare traffico nella rete, in modo da impegnare il canale per avere condizioni simili ad un tipico scenario di utilizzo.
Per ogni dispositivo � interessante mostrare per cosa � stato utilizzato. Nel caso di un nodo della rete, mostreremo anche il sistema operativo, il kernel, il processore e scheda di rete.




\begin{paragraph}{ZyXEL NBG4615 v2}
Access point utilizzato durante i test. � stata impostata la modalit� Wi-Fi 802.11b/g/n.

\end{paragraph}


\begin{paragraph}{LB-LINK}
Adattatore Wi-Fi USB. Velocit� fino a 150Mb/s

\end{paragraph}



\begin{paragraph}{HP Pavilion dv6 Entertainment PC}
Questo notebook � stato utilizzato come client. Abbiamo montato un kernel versione 4.0.1, modificato tramite la procedura illustrata precedentemente. 
Le specifiche tecniche sono: 
\begin{itemize}
 \item Kernel: Linux versione 4.0.1 modificata
 \item Processore: Intel Core i5 CPU M 430 @ 2.27GHz x 4 
 \item Sistema operativo: Ubuntu 14.04 LTS 64-bit
 
\end{itemize}

\end{paragraph}

\begin{paragraph}{Raspberry Pi Model 2}
Abbiamo utilizzato questo raspberry per creare traffico sulla rete.
Le specifiche tecniche sono: 
\begin{itemize}
 \item Kernel: Linux versione 3.18.0-20-rpi2  
 \item Processore: 900MHz quad-core ARM Cortex-A7 CPU 
 \item Sistema operativo: Raspbian Wheezy
 
\end{itemize}

\end{paragraph}


\begin{paragraph}{UDOO Quad}
Abbiamo utilizzato questo raspberry per creare traffico sulla rete.
Le specifiche tecniche sono: 
\begin{itemize}
 \item Kernel: 
 \item Processore: Freescale i.MX 6 ARM Cortex-A9 CPU Dual/Quad core 1GHz 
 \item Sistema operativo: Raspbian Wheezy
 
\end{itemize}

\end{paragraph}



\begin{paragraph}{Raspberry Pi Model B}
Abbiamo utilizzato questo raspberry per creare traffico sulla rete.
Le specifiche tecniche sono: 
\begin{itemize}
 \item Kernel: 
 \item Processore: 
 \item Sistema operativo: Raspbian Wheezy
 
\end{itemize}

\end{paragraph}


\end{section}




\begin{section}{Parametri di valutazione}
Per poter analizzare i test in modo ottimale e per avere dei dati su cui lavorare abbiamo deciso di controllare alcuni parametri.
\\
I parametri che ci interessano maggiormente sono:

\begin{itemize}
 \item Id: l'Id del pacchetto inviato.
 \item ACK: se � stato ricevuto un ACK o un NACK da parte dell'AP. 
 \item Data: � dato dalla data e dall'orario di invio del pacchetto.
 \item Tempo: � il tempo in millisecondi tra l'invio del pacchetto e la ricezione della notifica da parte dell'access point.
 \item {Retry count}: � il numero di tentativi di invio di un determinato pacchetto
 \item {Versione IP}: IPv4 o IPv6
 \item Configurazione: � la configurazione dei dispositivi utilizzati durante l'esperimento.
 \item Wait: indica se la recv � bloccante.
 
\end{itemize}
Abbiamo scelto questi parametri perch� ci permettono di poter giudicare in maniera chiara l'andamento dei pacchetti e la situazione della rete.
In particolare sono molto significativi il tempo, l'ACK ed il retry count.
Grazie a questi dati si pu� analizzare in modo dettagliato la situazione di ogni singolo pacchetto. L'applicazione pu� leggere l'ACK e successivamente decidere di rimandare il pacchetto in base ai millisecondi passati prima di ricevere la notifica.
Il numero di retry count risulta rilevante per confrontare differenti situazioni di traffico, oppure per notare cosa succede in caso di trasmissione in movimento.
Gli altri parametri che abbiamo deciso di utilizzare hanno un valore pi� trascurabile per un singolo pacchetto, ma possono diventare eloquenti per analizzare i dati a posteriori.
In particolare si potrebbe notare in base all'orario se c'� un evidente rallentamento della trasmissione. Ad esempio si potrebbe notare come in una zona industriale la QoS migliori durante la sera/notte.
Un altro dato che pu� essere utilizzato per esaminare i dati raccolti � la versione IP, si pu� controllare se c'� un differenza notevole tra IPv6 e IPv4 a parit� di condizioni.
Le configurazioni, invece, riguardano i dispositivi utilizzati durante un test e lo scenario applicativo. Andremo a mostrare quali configurazioni sono state provate in modo pi� dettagliato successivamente.
Per quanto riguarda la wait abbiamo deciso di fare sia una recv bloccante che una non bloccante. Abbiamo fatto dei test con entrambe e abbiamo analizzato le differenze, che andremo a descrivere pi� avanti.

Si potrebbero utilizzare anche altre informazioni ( ad esempio la bitrate ) per analizzare meglio i risultati, che saranno approfondite negli sviluppi futuri.








\end{section}

\begin{section}{Configurazioni}
Per ottenere dei risultati che potessero rispecchiare un reale utilizzo da parte di un nodo mobile abbiamo creato diverse configurazioni di dispositivi.
In particolare abbiamo combinato l'utilizzo di uno o pi� dispositivi 
\end{section}

    

\begin{section}{Risultati}

Andiamo ora ad analizzare i risultati ottenuti.
\end{section}




\end{chapter}

\end{document}
